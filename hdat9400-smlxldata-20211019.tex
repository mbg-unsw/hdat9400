% HDAT9400 Data Management: S, M, L, XL Data
% (c) 2021 Malcolm Gillies <malcolm.gillies@unsw.edu.au>
% https://github.com/mbg-unsw/hdat9400
%
% This work is licensed under a
% Creative Commons Attribution-NonCommercial-ShareAlike 4.0
% International Licence
\documentclass[aspectratio=169,12pt,usepdftitle=false]{beamer} % XXXX fix AR here
\usepackage[latin1]{inputenc}
\usepackage[T1]{fontenc}
\usepackage{textcomp}
\usefonttheme{serif} % need this with Charter font
\usetheme{Berlin}  % using default now
\usecolortheme{beaver}  % using default now
\usepackage[libertine]{libertine} % not using osf (old-style figures)
\usepackage[scale=0.9]{tgheros} % scale to match libertine
\usepackage[varqu,varl]{inconsolata}
\usepackage[libertine]{newtxmath}
\usepackage{graphicx}
\usepackage{tikz}
\usepackage{tikzpagenodes}
\usepackage[round]{natbib}
\usepackage{gitinfo2}

\hypersetup{pdfauthor={(C) 2021 Malcolm Gillies. CC-BY-NC-SA},
            pdftitle={HDAT9400 Data Management: S, M, L, XL Data},
	    pdfkeywords={data management, time complexity, data processing, health data, big data}}

\renewcommand{\gitMark}{\color{gray}\texttt{\tiny\gitBranch\,@\,\gitAbbrevHash\,\gitAuthorDate}}

\renewcommand{\bibsection}{} % suppress "References" section

\setbeamertemplate{navigation symbols}{} % remove navigation symbols
\setbeamertemplate{page number in head/foot}[totalpagenumber]
\setbeamercolor*{item}{fg=darkred}

\title{HDAT9400 Data Management: S, M, L, XL Data}
\institute{\url{https://github.com/mbg-unsw/hdat9400}}
\author{Malcolm Gillies}
\date{19 October 2021}
\usebackgroundtemplate{%
\begin{tikzpicture}[remember picture,overlay]
    \node[anchor=south west,scale=1,rotate=90] at ([shift={(0cm,0cm)}]current page marginpar area.south east) {\gitMark};
\end{tikzpicture}%
}

\titlegraphic{%
\begin{tikzpicture}[overlay,remember picture]
	\node[anchor=south west,scale=0.33] at ([shift={(0cm,0cm)}]current page text area.south west)
	{\includegraphics{ref/unsw-cbdrh-land}};
\end{tikzpicture}
}

\begin{document}

{
%\usebackgroundtemplate{}
\begin{frame}
\titlepage
\end{frame}
}

% -	Why care about data size?
% o	Time and space are finite
% o	Costs and deadlines
% o	Work smarter not harder

% -	How big are typical health data sets?
% o	Examples: rows x columns
% *** check example data used in this course incl. assignments
% RCT results
% Clinical registry
% Population survey (NHANES)
% PBS data
% ADPC
% EHR data
% (Clickstream)
%
% Data	rows		cols	GB
% RoCC	1.50E+04	25	0.0016875
% PDC	1.60E+06	170	1.224
% APDC	5.30E+07	64	15.264
% PBS	4.00E+09	22	396
%
% -	What are the (performance) characteristics of typical data processing methods?
% Data items (order), data items per second, Complexity, Flexibility, Cost/infrastructure requirements
% o	In memory [R]
% o	On disk sequential scan [SAS]
% o	OLTP [RDBMS]
% o	OLAP [DuckDB]
% o	Parallel [Spark]
%
% -	What size data can you analyse with R, SAS, or an RDBMS?
%       Do some experiments and show results
%
% -	Centre for Modestly Sized Data Research in Health
%
% -	Algorithms, efficiency and optimisation
% o	Big O notation
% o	Sorting, Indexes, hashing
% o	SQL and query optimisation
%
% -	Statistical modelling with bigger data
%
% -	Final real-world examples
% o	E.g. BHI hospital performance reporting, …
%


% About me
% - first degree computer science and pharmacology
% - worked in IT, academic research, non-profit and government
% - currently Senior Biostatistician with Medicines Policy Research Unit @ CBDRH
% - data, statistics and pharmacoepidemiology
% - SAS, R, databases and software engineering ``Data Carpentry''

% Presentation outline

% Breaks and diversions

% Points for discussion/questions

% -	Pop Quiz 1
% o	Which of these datasets is the biggest?

% -	Pop Quiz 2
% o	

% -	Potential diversion: SPDY

% -	Potential diversion: TP and Sabre

% Backgrounds: Rem Koolhaas buildings? (credits page)
\begin{frame}{Thanks}
    \begin{itemize}
        \item XXXX
    \end{itemize}
\end{frame}

\begin{frame}{References}
        \tiny\bibliography{hdat9400.bib}
        \bibliographystyle{abbrvnat}
\end{frame}

\end{document}
